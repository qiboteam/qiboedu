\documentclass[8pt, xcolor={svgnames}, hyperref={linkcolor=black}]{beamer}
\usepackage[labelfont={color=amethyst,bf}]{caption}
\setbeamercolor{background canvas}{bg=white}
\usetheme[progressbar=frametitle]{metropolis}
\usepackage{appendixnumberbeamer}
\usepackage{url}
\usepackage{booktabs}
\usepackage{braket}
\usepackage[scale=2]{ccicons}
\usepackage{amsfonts} 
\usepackage{amssymb}
\usepackage[english]{babel}
\colorlet{col1}{teal}
\colorlet{col2}{yellow}
\colorlet{col3}{green}
\usepackage{fontawesome}
\usepackage{subcaption}
\usepackage{multicol}
\usepackage{bm}
\usepackage{algorithm}
\usepackage{algpseudocode}
\usepackage{enumitem}

\usepackage[]{pseudo}


\usepackage{tikz}
\usetikzlibrary{positioning,arrows,calc,math,angles,quotes}
\usepackage{blochsphere}

\usetikzlibrary{arrows,automata}
\usetikzlibrary{positioning}
\usetikzlibrary{arrows.meta,
                bending,
                intersections,
                quotes,
                shapes.geometric}

\tikzset{
    state/.style={
           rectangle,
           rounded corners,
           draw=black, very thick,
           minimum height=1em,
           inner sep=2pt,
           text centered,
           },
}


\definecolor{myv}{rgb}{0.36, 0.22, 0.33}
\definecolor{gio}{rgb}{0.45, 0.31, 0.59}
\definecolor{light}{rgb}{0.8, 0.8, 1}
\definecolor{warmblack}{rgb}{0.0, 0.26, 0.26}
\definecolor{brown(web)}{rgb}{0.65, 0.16, 0.16}
\definecolor{cadmiumgreen}{rgb}{0.0, 0.42, 0.24}
\definecolor{darkmidnightblue}{rgb}{0.0, 0.2, 0.4}
\definecolor{brightube}{rgb}{0.82, 0.62, 0.91}
\definecolor{bleudefrance}{rgb}{0.19, 0.55, 0.91}
\definecolor{brightmaroon}{rgb}{0.76, 0.13, 0.28}
\definecolor{codegreen}{rgb}{0,0.6,0}
\definecolor{codegray}{rgb}{0.5,0.5,0.5}
\definecolor{codepurple}{rgb}{0.58,0,0.82}
\definecolor{backcolour}{rgb}{0.95,0.95,0.92}
\definecolor{amethyst}{rgb}{0.6, 0.33, 0.73}

\definecolor{light-gray}{gray}{0.95}
\newcommand{\code}[1]{\colorbox{light-gray}{\texttt{#1}}}


\usepackage{listings}
\lstdefinestyle{mystyle}{
    backgroundcolor=\color{backcolour},   
    commentstyle=\color{codegreen},
    keywordstyle=\color{codepurple},
    numberstyle=\tiny\color{codepurple},
    stringstyle=\color{magenta},
    basicstyle=\footnotesize,
    breakatwhitespace=false,         
    breaklines=true,                 
    captionpos=b,                    
    keepspaces=true,                 
    numbers=left,                    
    numbersep=5pt,                  
    showspaces=false,                
    showstringspaces=false,
    showtabs=false,                  
    tabsize=2
}

\lstset{style=mystyle}
\usepackage[most]{tcolorbox}
\usepackage{xcolor}


%\usepackage[citecolor = green, linkcolor = blue, bookmarks=true, urlcolor=blue,
%colorlinks=true, pagebackref=true]{hyperref}


%\usepackage{xspace}

\title{Introduction to Quantum Computing}
\subtitle{Quantum Computing Minicourse}
\date{8 April 2024}
\author[Stefano Carrazza and Matteo Robbiati]{Stefano Carrazza and Matteo Robbiati}
\titlegraphic{
\begin{tikzpicture}[overlay, remember picture]

\node[at=(current page.south east), anchor=south east] {%
\includegraphics[width=.18\textwidth]{figures/qibo.png} 
\includegraphics[width=.18\textwidth]{figures/unimi.png} 
\includegraphics[width=.18\textwidth]{figures/cern.png}  
\includegraphics[width=.18\textwidth]{figures/ictp.png}  
};
\end{tikzpicture}
}


\begin{document}

\maketitle

\begin{frame}{Compute quantum mechanics}
\pause
  \begin{itemize}[noitemsep]
  \item<2,3>[\faGear] Representing $N$ particles is difficult;
  \item<3>[\faGears] considering $N$ spins ($\uparrow, \downarrow$), we deal with a $2^N$ dimensional Hilbert space!
  \end{itemize}
  \vspace{0.5cm}
  \begin{multicols}{2}
    \begin{figure}
       \uncover<2,3>{\includegraphics[width=0.45\textwidth]{figures/spins.png}}%
    \end{figure}
    \begin{figure}
       \uncover<3>{\includegraphics[width=0.45\textwidth]{figures/hilb.png}}%
    \end{figure}
\end{multicols}
\end{frame}

\begin{frame}{What can we do?}
\pause
\begin{itemize}[noitemsep]
\item[1.] we can try to use classical methods to represent the system;
\pause
\item[2.] we can build a quantum computer.
\end{itemize}
\pause
\begin{figure}
   \includegraphics[width=0.4\textwidth, height=0.55\textheight]{figures/feynmann.jpg}%
   $\,\,$ \pause
   \includegraphics[width=0.4\textwidth, height=0.55\textheight]{figures/qcomp.png}
\end{figure}

\small
\textit{Nature isn't classical, dammit, and if you want to make a simulation of nature, 
you'd better make it quantum mechanical, and by golly it's a wonderful problem, 
because it doesn't look so easy.} 

\href{https://link.springer.com/article/10.1007/BF02650179}{\faBook\,\, Richard Feynman, 1982, Simulating Physics with Computers}
\end{frame}

\begin{frame}{What can we do?}
\begin{itemize}[noitemsep]
\item[1.] \textbf{\textcolor{amethyst}{we can try to use classical methods to represent the system}};
\item[2.] we can build a quantum computer.
\end{itemize}
\begin{figure}
   \includegraphics[width=0.4\textwidth, height=0.55\textheight]{figures/feynmann.jpg}%
   $\,\,$ 
   \includegraphics[width=0.4\textwidth, height=0.55\textheight]{figures/qcomp.png}
\end{figure}

\small
\textit{Nature isn't classical, dammit, and if you want to make a simulation of nature, 
you'd better make it quantum mechanical, and by golly it's a wonderful problem, 
because it doesn't look so easy.} 

\href{https://link.springer.com/article/10.1007/BF02650179}{\faBook\,\, Richard Feynman, 1982, Simulating Physics with Computers}
\end{frame}

\begin{frame}{Can we represent a state with a classical computer?}
Let's suppose we want to represent a system of \textbf{qubits} ($\uparrow, \downarrow$).
\pause
\begin{itemize}[noitemsep]
\item[1.] Each \texttt{float 64} requires 8 bytes of memory to be stored;
\pause
\item[2.] each \texttt{complex 128} requires 16 bytes of memory;
\pause
\item[3.] let's take a nice 32Gb of RAM: it can store up to $2$ billions of \texttt{complex 128}.
\pause
\item[4.] a $30$ qubits state requires $\sim 1$ billion of complex numbers;
\pause
\item[5.] a $31$ qubits state cannot be represented by my PC;
\pause
\item[6.] no problem. Let's get serious: \texttt{Fugaku}!
\end{itemize}

\begin{figure}
   \includegraphics[width=0.48\textwidth, height=0.4\textheight]{figures/fugaku.jpeg}%
   $\,\,$
   \pause
   \includegraphics[width=0.48\textwidth, height=0.4\textheight]{figures/sad_fugaku.jpg}
\end{figure}
\end{frame}

\begin{frame}{Some smart strategies}
   \begin{itemize}[noitemsep]
      \item<2,3,4>[1.] Variational Monte Carlo (VMC): given a wave function $\Psi(\bm{x}|\bm{\theta})$ and a target $H$, MC methods are used to minimize:
      $$ \frac{\int \text{d}\bm{x}\, \Psi^*(\bm{x}|\bm{\theta})\, H \, 
      \Psi(\bm{x}|\bm{\theta})}{\int \text{d}\bm{x}\, |\Psi(\bm{x}|\bm{\theta})|^2}\geq E_0; $$
      \item<3,4>[2.] Tensor Networks (TNs): contraction of complex systems into simpler structures;
      \item<4>[3.] Neural Network Quantum States: use complex ANNs to represent the state.
   \end{itemize}
   \begin{multicols}{3}
   \begin{figure}
      \uncover<2,3,4>{\includegraphics[width=0.9\linewidth, height=0.35\textheight]{figures/dices.png}
      \caption*{\href{https://arxiv.org/abs/1508.02989}{\faBook\,\,arXiv:1508.02989}}}%
      \uncover<3,4>{\includegraphics[width=0.9\linewidth, height=0.35\textheight]{figures/tn.png}
      \caption*{\href{https://arxiv.org/abs/1708.00006}{\faBook\,\,arXiv:1708.00006}}}%
      \uncover<4>{\includegraphics[width=0.9\linewidth, height=0.35\textheight]{figures/bm.png}
      \caption*{\href{https://arxiv.org/abs/1606.02318}{\faBook\,\,arXiv:1606.02318}}}
   \end{figure}   
   \end{multicols}
\end{frame}

\begin{frame}{Is this enough? \hfill \href{https://arxiv.org/abs/2103.10293}{\faBook\,\,arXiv:2103.10293}}
\pause
\begin{figure}
   \includegraphics[width=0.7\textwidth]{figures/complexity.jpg}
\end{figure}
\end{frame}

\section{A snapshot of quantum computing}

\begin{frame}{Qubits}
        \begin{itemize}[noitemsep]
           \item<2,3,4,5>[1.] classical bits are replaced by \textbf{qubits}
           $ \ket{\psi} = \alpha\ket{0}+\beta\ket{1}$.
           \item<3,4,5>[2.] we can manipulate the qubit state applying \textbf{gates}: $\ket{\psi'}=\mathcal{U}(\bm{\theta})\ket{\psi}.$

           Typically we use 1-qubit and 2-qubits gates!
         %   $$ H \ket{0} = \frac{\ket{0}+\ket{1}}{\sqrt{2}} \qquad \text{with}\qquad  H = 
         %   \begin{pmatrix}
         %   1 & 1 \\ 1 & -1
         %   \end{pmatrix}.
         %   $$
           \item<4,5>[3.] combine together gates to build \textbf{quantum circuits};
           \item<5>[4.] to access the information we need to measure the system.
        \end{itemize}
        \begin{figure}
            \only<2>{\includegraphics[width=0.95\linewidth, height=0.45\textheight]{figures/qubits.png}}%
            \only<3>{\includegraphics[width=0.5\linewidth, height=0.45\textheight]{figures/pulses.png}}%
            \only<4>{\includegraphics[width=0.9\linewidth, height=0.45\textheight]{figures/circuit.png}}
            \only<5>{\includegraphics[width=0.7\linewidth, height=0.45\textheight]{figures/measurement.png}}
        \end{figure}        
\end{frame}


\begin{frame}{New computational power: an example}
\pause
With quantum computing, we introduce new tools.
\pause
\begin{itemize}[noitemsep]
\item[\faRocket] prepare a quantum state in the computational zero $\ket{0}$;
\pause
\item[\faSliders] we can prepare superposition: 
$$H\ket{0} = \frac{1}{\sqrt{2}}(\ket{0} + \ket{1}) \pause \quad \text{with} \quad H = \frac{1}{\sqrt{2}}
\begin{bmatrix} 1 & 1 \\ 1 & -1 \end{bmatrix},\,\, \ket{0}=\begin{bmatrix} 1 \\ 0 
\end{bmatrix},\,\,\ket{1}=\begin{bmatrix} 0 \\ 1 \end{bmatrix};$$
\pause
\item[\faShareAlt] let's apply a controlled-NOT (CNOT) gate on a second qubit prepared in $\ket{0}$:
$$ \text{CNOT} \biggl( \underbrace{\frac{1}{\sqrt{2}}(\ket{0} + \ket{1})}_{\text{control}}\otimes 
\ket{0} \biggr) = \pause \frac{1}{\sqrt{2}}(\ket{\textcolor{bleudefrance}{0}0} + 
\text{NOT}_{\rm targ}\ket{\textcolor{brightmaroon}{1}0}) = \pause \frac{1}{\sqrt{2}}(\ket{00} + \ket{11}). $$
\end{itemize}
\pause
\begin{figure}
   \includegraphics[width=0.6\linewidth]{figures/baby3.pdf}
\end{figure}    
\end{frame}

\begin{frame}{Parametric gates prepare variational quantum states}
\pause
\begin{itemize}[noitemsep]
\item[\faLightbulbO] Among the gates, parametric ones can be useful!
\pause
\item[\faMapPin] Let's consider a single qubit system:
\begin{equation*}
   \ket{\psi} = \alpha\ket{0}+\beta\ket{1} \qquad \pause \text{with} 
   \qquad \alpha = \cos{\frac{\theta}{2}}, \quad \beta = e^{i\phi}\sin{\frac{\theta}{2}}.
\end{equation*}
\end{itemize}
\begin{multicols}{2}
\def\rotationSphere{-110}
\def\radiusSphere{2cm}
\def\psiLat{45}
\def\psiLon{45}
\begin{blochsphere}[radius=\radiusSphere,opacity=0,rotation=\rotationSphere]
  % \drawBallGrid[style={opacity=.3}]{30}{45}
  % Draw the sphere...
  \drawLongitudeCircle[]{\rotationSphere}

  \drawLatitudeCircle[style={dashed}]{0}
  % Define the different points on the bloch sphere
  \labelLatLon{ket0}{90}{0};
  \labelLatLon{ket1}{-90}{0};
  \labelLatLon{ketminus}{0}{180};
  \labelLatLon{ketplus}{00}{0};
  \labelLatLon{ketpluspi2}{0}{-90};  % Longitude seems to be defined in the "wrong" direction, hence the minus
  \labelLatLon{ketplus3pi2}{0}{-270};
  \labelLatLon{psi}{\psiLat}{-\psiLon};
  % Draw and label the axis
  \draw[-latex] (0,0) -- (ket0) node[above,inner sep=.5mm] at (ket0) {\footnotesize $z$};
  \draw[-latex] (0,0) -- (ketplus) node[below,inner sep=.5mm] at (ketplus) {\footnotesize$x$};
  \draw[-latex] (0,0) -- (ketpluspi2) node[below,inner sep=.5mm] at (ketpluspi2) {\footnotesize $y$};
  % Draw |psi>
  \draw[-latex] (0,0) -- (psi) node[above]{\footnotesize $\ket{\psi}$};

  % Draw the angles
  \coordinate (origin) at (0,0);
  {
    % Will draw the angle/projection one the equatorial plane
    \setDrawingPlane{0}{0}
    % Draw the projection: cos is used to compute the length of the projection
    \draw[current plane,dashed] (0,0) -- (-90+\psiLon:{cos(\psiLat)*\radiusSphere}) coordinate (psiProjectedEquat) -- (psi);
    % Draw the angle
    \pic[current plane, draw,fill=purple!50,fill opacity=.5, text opacity=1,"\footnotesize $\phi$", angle eccentricity=2.2]{angle=ketplus--origin--psiProjectedEquat};
  }
  { \setLongitudinalDrawingPlane{\psiLon}
    % Draw the angle
    \pic[current plane, draw,fill=purple!50,fill opacity=.5, text opacity=1,"\footnotesize $\xi$", angle eccentricity=1.5]{angle=psi--origin--ket0};
  }
\end{blochsphere}

\pause

\def\rotationSphere{-110}
\def\radiusSphere{2cm}
\def\psiLat{45}
\def\psiLon{45}
\def\psiLatPrime{15} % Adjusted latitude for \psi' to be at the top
\def\psiLonPrime{-15} % Adjusted longitude for \psi' to be on the left-top

\begin{blochsphere}[radius=\radiusSphere, opacity=0, rotation=\rotationSphere]
  \drawLongitudeCircle[]{\rotationSphere}
  \drawLatitudeCircle[style={dashed}]{0}

  \labelLatLon{ket0}{90}{0};
  \labelLatLon{ket1}{-90}{0};
  \labelLatLon{ketminus}{0}{180};
  \labelLatLon{ketplus}{00}{0};
  \labelLatLon{ketpluspi2}{0}{-90};
  \labelLatLon{ketplus3pi2}{0}{-270};
  \labelLatLon{psi}{\psiLat}{-\psiLon};
  \labelLatLon{psiPrime}{\psiLatPrime}{-\psiLonPrime};

  \draw[-latex] (0,0) -- (ket0) node[above,inner sep=.5mm] at (ket0) {\footnotesize $z$};
  \draw[-latex] (0,0) -- (ketplus) node[below,inner sep=.5mm] at (ketplus) {\footnotesize$x$};
  \draw[-latex] (0,0) -- (ketpluspi2) node[below,inner sep=.5mm] at (ketpluspi2) {\footnotesize $y$};
  \draw[-latex, purple] (0,0) -- (psi) node[right, black]{\footnotesize $\ket{\psi}$};
  \draw[-latex, purple] (0,0) -- (psiPrime) node[left, black]{\footnotesize $\ket{\psi'}$};


% Draw modified trajectory with two curves before reaching psi'
\draw[purple, thick, ->] (psi) to[out=110, in=20] ++(-1,0.5) to[out=200, in=60] ++(0.3,-0.3) to[out=240, in=100] ++(-0.5,-0.3) to[out=280, in=100] (psiPrime);
\node[purple] at (-1,1) {$\mathcal{U}(\theta)$};
\end{blochsphere}
\end{multicols}
We can use as parametric gates the rotation around the axis of the block sphere:
$$ R_k(\theta) = \text{exp}\bigl[-i \theta \sigma_k\bigr], \qquad \text{with} \qquad 
\sigma_k \in \{I, \sigma_x, \sigma_y, \sigma_z\}. $$
\end{frame}

\begin{frame}{But can this work?}
\pause
\begin{itemize}[noitemsep]
\item[1.] Depend on the problem\pause, e.g. playing in the Hilbert space 
makes sense when tackling a many body problem!
\pause
\item[2.] We can take some more rational proof of utility.
\end{itemize}
\pause
Using Variational Quantum Circuits we define a Variational Quantum Computer!
\pause
\begin{multicols}{2}
\texttt{\\}
\begin{itemize}
\item<7,8,9>[1.] we want a quantum circuit $\mathcal{U}(\bm{\theta})$ to approximates some law $V$;
\item<8,9>[2.] executing $\mathcal{U}(\bm{\theta})$ we use a variational quantum state
to reach the solution;
\item<9>[3.] \textbf{Solovay-Kitaev theorem}: the number of gates needed by $\mathcal{U}$ to 
represent $V$ with precision $\delta$ is $\mathcal{O}(\log^c \delta^{-1})$, where
$c<4$.
\end{itemize}
\begin{figure}
    \uncover<6,7,8,9>{\includegraphics[width=0.5\textwidth]{figures/variational.png}}
\end{figure}
\end{multicols}
\end{frame}

\section{\texttt{Qibo} as full-stack playground}

\begin{frame}{Simulation, control and calibration with \texttt{Qibo}}
\pause
\begin{figure}
   \includegraphics[width=1\textwidth]{figures/qibo_ecosystem_webpage.pdf}
\end{figure}
\end{frame}

\end{document}
